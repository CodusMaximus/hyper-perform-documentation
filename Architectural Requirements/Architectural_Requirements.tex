\documentclass[11pt,a4paper]{article}

\usepackage{titling}
\usepackage[hidelinks]{hyperref}
\usepackage{graphicx}
\usepackage{grffile}
\usepackage{float}
\usepackage{geometry}

\newcommand{\subtitle}[1]{
	\posttitle{
		\par\end{center}
	\begin{center}\large#1\end{center}
	\vskip0.5em}
}

\begin{document}
	\begin{figure}
		\centering
		\includegraphics[height=150px]{../Images/CodusMaximus_logo.jpg}
	\end{figure}
	\begin{figure}
		\centering
		\includegraphics[height=60px]{../Images/logo}
		\vspace{-50px}
	\end{figure}
	
	\title{ \textbf{Architectural Requirements Specification}}
	\subtitle{ Organisation: \url{https://github.com/Hyperperform}}	
	\author{
		\textbf{Developers:} \\
		Rohan Chhipa		\emph{14188377}	\\
		Avinash Singh		\emph{14043778}	\\
		Jason Gordon		\emph{14405025}	\\
		Claudio Da Silva	\emph{14205892}	\\
	}
	
	\date{\textbf{Updated \today}}
	\maketitle
	\begin{center}
		\textbf{Client}
		\begin{figure}[H]
			\begin{center}
				\includegraphics[width=0.20\columnwidth]{../Images/magna3}\\
				MagnaBC\\
				\url{http://www.magnabc.co.za/}
			\end{center}
		\end{figure}
	\end{center}
	
	\thispagestyle{empty}
	\pagebreak
	
	\tableofcontents
	\pagebreak


\section{Software Architecture Scope}
\begin{figure}[H]
	\begin{center}
		\includegraphics[scale=0.4]{../Images/Architecture_Overview.jpg}
		\caption{A high level break-down of the architecture to be used}
	\end{center}
\end{figure}

\pagebreak

An explanation of some of the adapters follow:
\begin{itemize}
	\item The EntrySystemsAdapter, provides an interface which various entry systems such as card readers, biometric scanners and the like, can map onto.
	\item The VersionControlAdapter, provides an interface which various version control systems such as GitHub, BitBucket and GitLab can map onto.
	\item The CalenderingSystemAdaoter, provides an interface for various calendering systems such as Google Calender, or Outlook calender to map onto.
	\item The ContinuousIntegrationAdapter, provides an interface which various Continuous Integration systems, such as Travis or Jenkins, can map onto.
	\item The BugSystemAdapter, provides an interface for which bug tracking systems such as Waffle.io or Trello can map onto.\\
\end{itemize}
Extended release adapters will include:
\begin{itemize}
	\item The WearablesAdapter, which should provide an interface for wearable devices such as Fitbits and Samsung Gears can map on to.\\\\ 
\end{itemize}
Further information will be provided under integrations.

\section{Overall Software Architecture}

\subsection{Quality Requirements}
The following section will focus on the quality requirements of the overall system at its highest level of granularity, and all the requirements defined are defined for all components of the system.

\subsubsection{Flexibility}
The most important element of the system is its flexibility. Each component of the system is decoupled from every other component, thus allowing you to change certain components without affecting anything else. The main focus of the software is to source data from various integrations that change with the current market trends and business environment. Thus the need for additional integrations and replacement of default integrations will arise, allowing for a system fully embracing the principles of IOT (Internet of Things).\\ \\
The system is decoupled in the way that it is not limited to specific architectures, thus it should not be locked down to a specific provider for either the application server or the database.

\subsubsection{Maintainability}
Another very important aspect of the system is the maintainability of the system, to the extent that the system is easy for future contributors to change, maintain and further develop.
\begin{itemize}
	\item Future developers are able to easily understand the system, and modify it as needed provided that the specified contracts are adhered to.
	\item The technologies chosen for the system can be reasonably expected to be available for a long time as well as remain open source and available. However due to the front-end being completely decoupled from the back-end it would be easy to change the front-end technology without affecting the back-end functionality.
	\item The system can be maintained without having the provided services unavailable for large periods of time.
\end{itemize}

\subsubsection{Integrability}
The system allows for easy future integration requirements by providing access to its services using public standards. All external systems should have a preprocessor interface on which to communicate and map on to.

\subsubsection{Scalability}
The system should theoretically be usable in large industries of thousands of employees. Thus the system has been designed to handle the load and grow as needed. To accomplish this we make use of the message bus architecture.

\subsubsection{Reliability}
The system is reliable in the fact that many events will be constantly captured. All events will be persisted and used at a later stage thus leading to more accurate calculations.

\subsubsection{Security}
This is not the main focus of the system however for any system this requirement is important. The following security measures have been put in place:
\begin{itemize}
	\item Proper authentication is to be preformed in order to gain access to confidential user information from the database. 
	\item Outside components can only gain access through the provided REST endpoints.
\end{itemize}

\subsubsection{Auditability}
Logging occurs for:
\begin{itemize}
	\item Events captured into the queue.
	\item Access to the users personal information.
\end{itemize}
Logs contain at least for all different logs:
\begin{itemize}
	\item An id pertaining to user performing the task.
	\item A description of the activity happening.
	\item The time at which this activity happened.
\end{itemize}

\subsubsection{Usability}
The system has been created to be intuitive and easy to use, the system is not difficult for a user to learn and interact with. Some things that have been emphasised are:
\begin{itemize}
	\item Useful and helpful error messages, as well as client side validation as far as possible.
	\item Labels and hints where possible to guide users.
\end{itemize}

\subsubsection{Testability}
The system is testable through Continuous Integration using:  
\begin{itemize}
	\item Automated isolated tests using mocks
	\item Automated integration tests using the actual realizations 
\end{itemize}
Tests are run ensure that all pre and post-conditions are properly met and that the system is in working order.

\subsubsection{Deployability}
\begin{itemize}
	\item The system must be built using only the build tool and scripts.
	\item The system is be deployable on cross platform environments.
	\item The system is easily adaptable to different environments with different databases and message brokers.
	\item The system has been Dockerised, allowing for easy distribution. 
\end{itemize}

\subsection{Architectural Constraints}
The following constraints are non-negotiable and must be met for the system to properly meet client requirements:
\begin{itemize}
	\item The system is platform independent, allowing for docker to handle the environment setup.
	\item All software used in the creation of this project, including frameworks, libraries and application servers must be strictly open source, as well as preferably well supported and long term.
	\item The web interface runs correctly on all Webkit enabled browsers.
\end{itemize}

\subsection{Integration and Access Channels}
In terms of access channels, the project as it stands focuses on two main forms of access:
\begin{itemize}
	\item The web-client component, which will make use of REST calls in order to interact with the back end server. The web client is written in AngularJS 1.0, and interacts with the server using the RESTEasy implementation of JAX-RS REST wrapping. All services are only exposed through REST wrapping. Thus leaving no other access channel to the services.
\end{itemize}
With regards to integration, the system has the following adapters on which to map external integrations on to :
\begin{itemize}
	\item The EntrySystemsAdapter, provides an interface which various entry systems such as card readers, biometric scanners, etc. can map onto.
	\item The VersionControlAdapter, provides an interface which various version control systems such as GitHub, BitBucket and GitLab can map onto.
	\item The CalenderingSystemAdaoter, provides an interface for various calendering systems such as Google Calender, or Outlook calender can map onto.
	\item The ContinuousIntegrationAdapter, provides an interface which various Continuous Integration systems, such as Travis or Jenkins, can map onto.
	
	\item Similarly the events that are retrieved will come through the REST wrapping designed specifically for the events. Each integration has its own REST URL to which it sends events.
\end{itemize}
Thus since the events come through the REST URL's we don't have to continuously poll these systems. Instead the separate systems send the events when ready, these are known as webhooks. Once the events are received they are processed and placed on a messaging queue.\\ \\
However certain integrations might not support webhooks. A recommended approach to this problem would be to create a separate event poller, which is independent from the HyperPerform system. The poller can then be used to simulate webhook technology by sending the events to the provided REST URL's. Thus the HyperPerform system resources are not wasted on event polling and consistency is kept amongst the event listeners.

\section{Architectural Design}
\subsection{Architectural Tactics}
At the system level of granularity, that being the highest level of granularity, we do not specify any architectural tactics in order to concretely address the quality requirements for the system.
\subsection{Architectural Patterns}
Some of the patterns used at the system level include:
\begin{itemize}
	\item The layered pattern, which limits access of component within one layer to components which are either in the same or the next lower level layer. A great strength it has is that a can be relatively easily replaced. In particular, one can add further access channels without changing any lower layers (A critical component of this system) within the software system and changing the persistence provider would only require changing the persistence API.
	\item The Message Bus pattern, an enterprise integration pattern which allows for safe and efficient handling of events via messages, using a response and request queue, one can ensure that events and messages never get lost even if part of a system fails, thus allowing a greater fail safe in the system.
\end{itemize}
\subsection{Reference Architectures}
The key architecture at the system level is the Enterprise Service Bus. In order to properly network all devices in a proper IOT (Internet of Things) manner, one would implement a messaging queue system. We chose ActiveMQ Artemis as our realization of this pattern, as it integrates quite nicely with JavaEE, especially the WildFly server we are running, and has decent support, as well as many features we can make use of.

\pagebreak

%\section{Mobile Access}
%It was chosen to use Android Application Framework as the go to mobile access for this project, the reason being that:
%\begin{itemize}
%	\item Android is the most popularly used mobile OS in the market today
%	\item Android has less strings attached, and therefore makes it easier to design, not requiring specific environments in which to build, and has a higher cost for application maintenance.
%	\item Many open-source frameworks are available from the community, and many plug-ins and intents are available to make coding easier.
%	\item Android provides easier IOT(Internet of Things) structure, via their Samsung Hub, allowing for better integration with other devices later on.\\
%\end{itemize}
%The build tool to be used for Android will be Gradle, integrated with a Maven repository via Maven dependencies in the Gradle build file.
%
%\pagebreak
	
\section{Technologies}
\subsection{Back end}
\begin{itemize}
	\item WildFly as the JavaEE container and server.
	\item JUnit as a unit testing framework along with Spring.
	\item RESTEasy implementation of JAX-RS for REST wrapping.
	\item ActiveMQ Artemis as the message broker for the system.
%	\item Scala for functional programming on mathematical algorithms.
\end{itemize}
\subsection{Front end}
\begin{itemize}
	\item HTML5, CSS/SCSS and SASS, Bootstrap and JQuery.
	\item AngularJS 1.0 is used to access the REST services.
%	\item BIRT as a front end reporting tool, mixed with 
	\item ChartJS to create appealing charts on the front-end.
%	\item Android Application Framework, for developing native Android applications.
\end{itemize}
\subsection{Persistence}
\begin{itemize}
	\item PostgreSQL for the database.
	\item JPA as an Object Relational Mapper.
\end{itemize}

\subsection{Build tools}
\begin{itemize}
	\item Maven for assembling war applications for the back end.
	\item Gulp for clustering code into one application for the front end.
\end{itemize}

\subsection{Deployment tools}
\begin{itemize}
	\item Docker as a shipping container.
	\item TravisCI as a Continuous Integration tool.
\end{itemize}

\end{document}